%
% uaThesis example (for a thesis written in Portuguese)
%
% the complete list of options and commands can be found in uaThesis.sty
%

\documentclass[11pt,twoside,a4paper]{report}
\usepackage[DETI,newLogo]{uaThesis}

\def\ThesisYear{2017}

% optional packages
\usepackage[portuguese]{babel}

\usepackage[T1]{fontenc}
\usepackage{hyperref}
\usepackage{amsmath}
\usepackage{amssymb}
\usepackage{xspace}% used by \sigla

% optional (comment to use default)s
%   depth of the table of contents
%     1 ... chapther and sections
%     2 ... chapters, sections, and subsections
%     3 ... chapters, sections, subsections, and subsubsections
\setcounter{tocdepth}{3}

% optional (comment to used default)
%   horizontal line to separate floats (figures and tables) from text
\def\topfigrule{\kern 7.8pt \hrule width\textwidth\kern -8.2pt\relax}
\def\dblfigrule{\kern 7.8pt \hrule width\textwidth\kern -8.2pt\relax}
\def\botfigrule{\kern -7.8pt \hrule width\textwidth\kern 8.2pt\relax}

% custom macros (could also be defined using \newcommand)
\def\I{\mathtt{i}}         % one possible way to represent $\sqrt{-1}$
\def\Exp#1{e^{2\pi\I #1}}  % argument inside braces, i.e., "{}"
\def\EXP#1.{e^{2\pi\I #1}} % argument finishes when a full stop is encountered, i.e., "."
\def\sigla{\LaTeX\xspace}  % use as "blabla \sigla blabla (no need to do "blabla \sigla\ blabla"

\def\AddVMargin#1{\setbox0=\hbox{#1}%
                  \dimen0=\ht0\advance\dimen0 by 2pt\ht0=\dimen0%
                  \dimen0=\dp0\advance\dimen0 by 2pt\dp0=\dimen0%
                  \box0}   % add extra vertical space above and below the argument (#1)
\def\Header#1#2{\setbox1=\hbox{#1}\setbox2=\hbox{#2}%
           \ifdim\wd1>\wd2\dimen0=\wd1\else\dimen0=\wd2\fi%
           \AddVMargin{\parbox{\dimen0}{\centering #1\\#2}}} % put #1 on top #2


\begin{document}

%
% Cover page (use only one of the first two \TitlePage)
%

% First alternative, with a figure
\TitlePage
  %\GRID  % for debugging ONLY
  \HEADER{\BAR\FIG{\includegraphics[height=60mm]{uaLogoOld}}} % the \FIG{} is optional
         {\ThesisYear}
  \TITLE{Henrique Gapo \newline Oliveira}
        {Plataforma de medida com suporte para mobilidade}
\EndTitlePage
\titlepage\ \endtitlepage % empty page



%
% Initial thesis pages
%

\TitlePage
  \HEADER{}{\ThesisYear}
  \TITLE{Henrique Gapo \newline Oliveira}
        {Plataforma de medida com suporte para mobilidade}
  \vspace*{15mm}
  \TEXT{}
       {Disserta\c c\~ao apresentada \`a Universidade de Aveiro para cumprimento dos requesitos
        necess\'arios \`a obten\c c\~ao do grau de Doutor em X, realizada sob a orienta\c c\~ao
        cient\'\i fica de Y, Professor do Departamento Z da Universidade de Aveiro}
\EndTitlePage
\titlepage\ \endtitlepage % empty page

\TitlePage
  \vspace*{55mm}
  \TEXT{\textbf{o j\'uri~/~the jury\newline}}
       {}
  \TEXT{presidente~/~president}
       {\textbf{ABC}\newline {\small
        Professor Catedr\'atico da Universidade de Aveiro (por delega\c c\~ao da Reitora da
        Universidade de Aveiro)}}
  \vspace*{5mm}
  \TEXT{vogais~/~examiners committee}
       {\textbf{DEF}\newline {\small
        Professor Catedr\'atico da Universidade de Aveiro (orientador)}}
  \vspace*{5mm}
  \TEXT{}
       {\textbf{GHI}\newline {\small
        Professor associado da Universidade J (co-orientador)}}
  \vspace*{5mm}
  \TEXT{}
       {\textbf{KLM}\newline {\small
        Professor Catedr\'atico da Universidade N}}
\EndTitlePage
\titlepage\ \endtitlepage % empty page

\TitlePage
  \vspace*{55mm}
  \TEXT{\textbf{agradecimentos~/\newline acknowledgements}}
       {\'E com muito gosto que aproveito esta oportunidade para agradecer a todos os que me
        ajudaram durante este longos e penosos anos, cheios de altos e baixos (mais baixos que
        altos)\ldots}
  \TEXT{}
       {Desejo tamb\'em pedir desculpa a todos que tiveram de suportar o meu desinteresse pelas
        tarefas mundanas do dia-a-dia, \ldots}
\EndTitlePage
\titlepage\ \endtitlepage % empty page

\TitlePage
  \vspace*{55mm}
  \TEXT{\textbf{Resumo}}
       {Nos dias que correm, \'e frequente um trabalho ser avaliado pela sua apar\^encia em vez de
        o ser pelo seu conte\'udo. Sendo assim, sem descurar este \'ultimo, nesta tese descrevemos
        maneiras revolucion\'arias de transformar um documento s\'olido e austero num documento
        s\'olido e belo, capaz de fazer chorar de alegria (ou de inveja) qualquer leitor, mesmo
        quando este n\~ao percebe nada do que l\'a est\'a escrito.}
  \TEXT{}
       {A explora\c c\~ao de novas descobertas na \'area da percep\c c\~ao visual, nomeadamente
        no que se refere \`a aprecia\c c\~ao de obras de arte geniais, \ldots}
\EndTitlePage
\titlepage\ \endtitlepage % empty page

\TitlePage
  \vspace*{55mm}
  \TEXT{\textbf{Abstract}}
       {Nowadays, it is usual to evaluate a work \ldots}
\EndTitlePage
\titlepage\ \endtitlepage % empty page


%
% Tables of contents, of figures, ...
%

\pagenumbering{roman}
\tableofcontents

\cleardoublepage
\listoffigures

\cleardoublepage
\listoftables


% The chapters (usually written using the isolatin font encoding ...)

\cleardoublepage
\pagenumbering{arabic}
\chapter{Introdu\c c\~ao}

Para este tipo de documentos, o autor prefere o estilo \verb+report+ ao estilo \verb-book-,
pelo que somente o primeiro \'e suportado oficialmente pelo ficheiro \verb:uaThesis.sty:.
\'E poss\'\i vel for\c car um novo cap\'\i tulo a come\c car numa p\'agina \'\i mpar atrav\'es
do uso do comando \verb!\cleardoublepage!. Deve-se sempre incluir a op\c c\~ao \verb+a4paper+
para especificar as dimens�es das folhas de papel.

Escusado ser\'a dizer (na realidade, escrever) que se a l\'\i ngua em que est\'a escrito o
documento n\~ao for o Ingl\^es, ser\'a preciso utilizar o ``pacote'' \verb.babel..


\section{Op\c c\~oes}

Apresentamos de seguida, uma lista das op\c c\~oes suportadas.
\begin{itemize}
  \item \verb+oldLogo+: usa o ``antigo'' logotipo da Universidade de Aveiro.
  \item \verb+newLogo+: usa o ``novo'' logotipo da Universidade de Aveiro.
  \item \verb+final+: \textbf{n\~ao escreve} o texto ``documento provis\'orio'' na capa: al\'em
        disso, todas as marcas que assinalam linhas demasiado compridas s\~ao eliminadas.
  \item \verb+DETI+, \verb+DM+, \verb+DF+: para teses escritas por alunos dos departamentos de
        electr\'onica, telecomunica\c c\~oes e inform\'atica, de matem\'atica, e de f\'\i sica.
        \'E muito f\'acil incluir uma op\c c\~ao para um outro departamentos editando o
        ficheiro \verb+uaThesis.sty+.
\end{itemize}


\chapter{Motores El�tricos}
O motor el�trico � uma m�quina que tem como fun��o converter energia el�trica em energia mec�nica. Hoje em dia, este equipamento tem um papel bastante importante na nossa sociedade, pois est� integrado em variad�ssimos instrumentos que facilitam o desenvolvimento de in�meras tarefas. 

Ultimamente presencia-se um crescimento no sector de transportes a utiliza��o do motor el�trico, devido � evolu��o tecnol�gica ao n�vel de baterias capazes de fornecer maior pot�ncia, maior autonomia e cargas menores, contribuindo desta forma para um mundo mais ecol�gico e sustent�vel, visto que o motor el�trico em contraste ao motor de combust�o apresenta maior efici�ncia energ�tica, n�o emite gases poluentes nocivos para a nossa atmosfera e � bastante silencioso.

\section{Breve Hist�ria}





\cleardoublepage
\chapter{Conclus\~ oes}

Que conclus\~oes?

Exemplo de duas entradas da ``\textit{bib file}'':

{\footnotesize
\begin{verbatim}
@Article
{
  Eliahou-1-1993-CLBNCL,
  author = {Eliahou, Shalom},
  title = {The $3x+1$ Problem: New Lower Bounds on Nontrivial Cycle Lengths},
  journal = {Discrete Mathematics},
  year = {1993},
  volume = {118},
  number = {1--3},
  pages = {45--56}
}

@Article
{
  Garner-1981-1-OCA,
  author = {Garner, Lynn E.},
  title = {On the Collatz $3n+1$ Algorithm},
  journal = {Proceedings of the American Mathematical Society},
  year = {1981},
  volume = {82},
  number = {1},
  pages = {19--22},
  month = May
}
\end{verbatim}
}


%
% The bibliography
%
\cleardoublepage
\iffalse
  % Use this is the final version
  %  unsrt produces numbered entries, sorted by order of citation
  %  plain produces numbered entries, sorted alphabetically
  %  other styles are possible (I recommend the harvard package)
  \bibliographystyle{unsrt}
  %\bibliographystyle{plain}
  \bibliography{my-bib-file}% replace by the name of name of your .bib file
\else
  % An example (the contents of the .bbl file)
  \begin{thebibliography}{10}

  \bibitem{Eliahou-1-1993-CLBNCL}
  Shalom Eliahou.
  \newblock The $3x+1$ problem: New lower bounds on nontrivial cycle lengths.
  \newblock {\em Discrete Mathematics}, 118(1--3):45--56, 1993.

  \bibitem{Garner-1981-1-OCA}
  Lynn~E. Garner.
  \newblock On the collatz $3n+1$ algorithm.
  \newblock {\em Proceedings of the American Mathematical Society}, 82(1):19--22,
    May 1981.
  \end{thebibliography}
\fi
\cleardoublepage

\end{document}
